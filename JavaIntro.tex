\documentclass[12pt,a4paper,twoside]{beamer}
\usepackage[T1]{fontenc}
\usepackage[utf8]{inputenc}
\usepackage[francais]{babel}
\usepackage{color}
\usepackage{graphicx}
\setbeamertemplate{blocks}[rounded][shadow=true]
\setbeamercolor{block title}{fg=black,bg=blue}
\setbeamertemplate{navigation symbols}{ \insertslidenavigationsymbol}\begin{document}
\author{Mathias DIDIER}\\
\title{Java Intro}
\maketitle
\begin{frame}
\frametitle{Plateforme}
\begin{block}{Organisation en packages}
-Entrées/sorties(java.io)\\
Reseaux (java.net)\\
-UI (java.awt/javax.swing)\\
-BDD (java.sql)\\
-Methode d'objet distant (java.rmi)\\
-Mathematique (java.math)\\
\end{block}
\end{frame}
\begin{frame}
\frametitle{Plateforme}
\end{frame}
\begin{frame}
\frametitle{Language Objet}
\begin{block}{Concept}
Modularité/Classes\\
Extensibilité/Heritage\\
\end{block}
\begin{block}{Technique}
-Classes,interfaces,paquetages\\
-Tout est en classe (Enveloppe pour les géneriques)\\
-Objets accessibles par references.\\
-Heritage simple des classes\\
-Heritage multiple des interfaces\\
-Polymorphisme\\
\end{block}
\end{frame}
\begin{frame}
\frametitle{Rappel}
\begin{block}{Encapsulation}
Implémente le masquage d'informations et la modularité\\
\end{block}
\begin{block}{Polymorphisme}
Permet a 2 objets d'obtenir un comportement different a un meme signal.\\
\end{block}
\begin{block}{Heritage}
On peut définir de nouvelles classes basées sur des classes existantes, pour obtenir du code réutilisable et de l'organisation. \\
envoyer des messages à des objets sans 
connaître au moment du codage leur type spécifique.\\
Liaisons dynamiques (def du dessus)\\
\end{block}
\end{frame}
\begin{frame}
\frametitle{Objet}
\begin{block}{Defini par }
Ses variables d'instance\\
Son comportement (methodes)\\
\end{block}
\begin{block}{Classe}
Constructeur qui defini l'objet (moule)\\
\end{block}
\begin{block}{This}
"this" fais reference a l'objet dans lequel le code ce situe\\
\end{block}
\end{frame}
\begin{frame}
\frametitle{Garbage Collector}
\begin{block}{Rammasse miettes}
Permet la desallocation memoire des objets\\
C'est automatique\\
on peut definir un comportement avant son execution en imlpementant la fontion \\
public void finalise()\\
\end{block}
\end{frame}
\begin{frame}
\frametitle{Casting}
\begin{block}{Upcating}
List list = new LinkedList();\\
\end{block}
\begin{block}{Downcating}
...(List) list; (cast classique)\\
\end{block}
\begin{block}{Nom}
Transtypage\\
\end{block}
\end{frame}
\begin{frame}
\frametitle{Controle d'acces}
\begin{block}{Static}
Une variable static est commune a tout les objets de cette classe\\
+- locales\\
Les methodes aussi peuvent etre static\\
\end{block}
\end{frame}
\begin{block}{Abstract}
Une classe abstraite ne peut pas crée directement d'objet\\
Permet de definir un etat par default, et un comportement generique\\
\begin{frame}
\frametitle{Tableau}
\end{block}
\begin{block}{Utilisation}
Les indices des tableaux commencent à 0\\
Taille : tableau.length\\
lève une ArrayIndexOutOfBoundsException si hors limites\\
\end{block}
\end{frame}
\begin{frame}
\frametitle{Chaines}
\begin{block}{Intro}
La classe String ne permet pas de modifier la chaine\\
La classe StringBuffer elle le permet\\
la méthode length retourne le nombre de caractère dans la chaîne\\
la méthode charAt retourne le caractère situé à la position donnée\\
indexOf(char ch) -> trouver la position de la premiere occurence de ch\\
indexOf(char ch, int start) : première position de ch >= start\\
\end{block}
\end{frame}
indexOf(String str) : première position de str\\
String substring(int begIndex, int endIndex)\\
\begin{frame}
\frametitle{Interfaces}
\begin{block}{Classe}
Classe abstraite, ou tout les methodes sont abstraites\\
Une classe peut implementer plusieurs interfaces\\
Une interface peut heriter d'une autre interface\\
\end{block}
\end{frame}
\begin{frame}
\frametitle{Exceptions}
\begin{block}{Intro}
Lancer, ou lever ou exception signifie envoier un signal particulier au bloc superieur\\
qui pourra la traiter, ou bien la renvoyer au bloc superieur\\
une classe succesptible de lever une exception doit le preciser dans son entete grace au mot throws\\
\end{block}
\end{frame}
\begin{frame}
\frametitle{}
\begin{block}{Quelques exceptions}
ArithmeticException -> division par zero\\
NullPointerException->reference nulle\\
ClassCastException -> tentative de cast illegal\\
NegativeArraySizeException -> creation tableau de taille negative\\
ArrayIndexOutOfBoundsException-> depassement limite d'un tableau\\
\end{block}
\end{frame}
\begin{frame}
\frametitle{Capturer/Lever}
\begin{block}{Capturer}
on execute le code dans un bloc try{}\\
on declare un catch(Exception e){}\\
qui traite l'exception recu\\
\end{block}
\begin{block}{Lever}
On utilise throw(Exception) pour lever une exception\\
\end{block}
\begin{block}{Finaly}
On peut declarer plusieur catch, et si aucun correspond, on declare un finally\\
\end{block}
\end{frame}
\begin{frame}
\frametitle{Creation exception}
\begin{block}{Type}
Error sont les throwable grave, il ne faut pas les capturer\\
Exceptions sont les exception courantes\\
\end{block}
\begin{block}{Redefinition}
On fait heriter de Throwable un objet qui doit se comporter comme une exception\\
On utilise la fonction getMessage pour modifier le message qui apparait si l'exception n'est pas capturée\\
\end{block}
\end{frame}
\begin{frame}
\frametitle{Collections}
\begin{block}{Intro}
Groupe d'elements uniques -> Collection\\
ou Set\\
Ensemble de paires clés-valeurs -> Map\\
(tableaux de hachages)\\
\end{block}
\end{frame}
\begin{frame}
\frametitle{List}
\begin{block}{LinkedList}
liste lié de voisins en voisins\\
optimisé pour des parcours complets\\
\end{block}
\begin{block}{ArrayList}
Optimisé pour les acces aléatoires\\
\end{block}
\begin{block}{List}
defini les methodes add, et remove\\
\end{block}
\end{frame}
\begin{frame}
\frametitle{Iterateur}
\begin{block}{Iterator}
On peut acceder aux elements grace a un iterateur\\
on le recupere grace a "list.iterator()"\\
on verifie qu'il y ai un element suivant avec "iterateur.hesNext()"\\
on saisi l'element suivant grace a "(Cast vers le type presque obligatore)iterateur.next()"\\
enleve le dernier objet saisi avec .remove\\
\end{block}
\end{frame}
\begin{frame}
\frametitle{Comparator}
\end{frame}
\begin{frame}
\frametitle{Comparator}
\begin{block}{Comparer 2 objets}
Pour pouvoir comparer 2 objets, il faut un comparateur aproprier\\
Ainsi la classe doit implementer l'interface comparable\\
\end{block}
\begin{block}{Fonctions}
int compare(Object o1,Object o2)\\
boolean equals(Object oth)\\
On a ainsi acces au fonction de base pour comparer, trier, etc\\
(Definiation d'un ordre naturel)\\
\end{block}
\end{frame}
\begin{frame}
\frametitle{Tiger}
\begin{block}{Amelioration des boucles}
for(Object element : liste){}\\
\end{block}
\begin{block}{Methodes}
Ajout des methodes a arguments variables\\
en fonction du type, et du nombre\\
pour nombre variable d'arg on utilise les "..."\\
ex "int ... valeurs"\\
\end{block}
\end{frame}
\begin{frame}
\frametitle{Scanner}
\begin{block}{Entree formattes}
Scanner s = new Scanner(System.in)\\
s.next()   s.nextInt etc\\
s.close()\\
\end{block}
\end{frame}
\begin{frame}
\frametitle{Liste D'elements avec type}
\begin{block}{Definition}
List<String> ls = new ArrayList<String>();\\
Ainsi plus besoin de caster lors de la recuperation d'un objet\\
\end{block}
\end{frame}
\end{document}
